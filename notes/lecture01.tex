%& -shell-escape_
\documentclass[12pt]{article}
\usepackage{infocourse}
%opening
\title{,,Теория информации``.\\ Лекция 1}
\author{А.В. Смаль}

\begin{document}

\maketitle

\section{Комбинаторный подход}
\subsection{Информация по Хартли}
Пусть задано некоторое конечное множество \(A\)~--- \emph{множество исходов}.
\begin{definition}[1928]
Определим \emph{количество информации в \(A\)} как \(\chi(A) = \log_2|A|\) (мы будем измерять количество информации в битах, поэтому все логарифмы будут по основанию \(2\), для байтов основание нужно было бы заменить на \(256\)).
\end{definition}

Если про некоторый \(x\in A\) стало известно, что \(x\in B\), то теперь для идентификации \(x\) нам достаточно \(\chi(A\cap B) = \log |A\cap B|\) битов, т.е. нам сообщили \(\chi(A) - \chi(A\cap B)\) битов информации.

\begin{example}
    Предположим, что мы хотим узнать некоторое неизвестное упорядочение множества $\{\seqn{a}{5}\}$. Нам стало известно,
    что \(a_1>a_2\) или \(a_3>a_4\). Сколько битов информации мы узнали? Множество \(A\) состоит из \(5!\) перестановок,
    множество \(B\)~--- из перестановок, которые удовлетворяют новому условию. Легко проверить, что \(|B| = 90\). Итого
    мы узнали \(\log 120 - \log 90 = \log(4/3)\) битов.
\end{example}

Пусть \(A\subset\binstr\times\binstr\). Обозначим через \(\pi_1(A)\) и \(\pi_2(A)\) проекции множества \(A\) на первую и вторую координату соответственно, а \(\chi_1(A) = \log|\pi_1(A)|\) и \(\chi_2(A) = \log|\pi_2(A)|\)~--- их сложность по Хартли.

\begin{theorem} 
\(\chi(A) \le \chi_1(A) + \chi_2(A)\).
\end{theorem}

\begin{definition}
Количество информации в второй компоненте \(A\) при известной первой
\[\chi_{2|1} = \log\left(\max_{a\in\pi_1(A)} |A_a|\right),\]
где $A_a = \{(a, x) \mid x\in \pi_2(A)\}$.
\end{definition}

\begin{theorem} 
\(\chi(A) \le \chi_1(A) + \chi_{2|1}(A)\).
\end{theorem}

\begin{theorem}\label{thm:volume}
Для \(A\subset\binstr\times\binstr\times\binstr\)
\[2\cdot\chi(A) \le \chi_{12}(A) + \chi_{13}(A) + \chi_{23}(A).\]
\end{theorem}
\begin{corollary}
Квадрат объёма трёхмерного тела не превосходит произведение площадей его проекций на координатные плоскости.
\end{corollary}

\begin{statement}
Если \(f: X\to Y\)
\begin{enumerate}
    \item является сюръекцией, то \(\chi(Y)\le \chi(X)\),
    \item является инъекцией, то \(\chi(X)\le \chi(Y)\).
\end{enumerate}
\end{statement}

\subsection{Применение: игра в 10 вопросов}
Сколько вопросов на ДА/НЕТ нужно задать, чтобы определить загаданное число от 1 до \(N\), если (a) можно задавать вопросы адаптивно; (б) вопросы нужно написать на бумажке заранее.

Оценка \(\lceil\log N\rceil \) достигается в обоих случаях, если задавать вопросы про биты двоичного представления загаданного числа.

Докажем нижнюю оценку. Пусть \(A=[N]\). Множество \(Q = \{(\seqn{q}{k})\}\)~--- множество протоколов (ответы на вопросы). 
Можно рассматривать \(A\) и \(Q\) как проекции некоторого множества исходов игры \(S\) на разные координаты. Тогда верны следующие неравенства:
\begin{itemize}
\item \( \chi_Q(S) = \chi(Q) \le \chi_1(Q) + \chi_2(Q) + \dotsb + \chi_k(Q) \le k, \)
\item \( \chi_A(S) = \chi(A) \le \chi(S) \le \chi_Q(S) + \chi_{A|Q}(S) \le k + 0 = k. \)
\end{itemize}
Таким образом получаем, что \(\log N = \chi(A) \le k\).

\subsection{Цена информации}
Пусть имеется некоторое неизвестное число от 1 до \(n\) (где \(n\ge2)\).
Разрешается задавать любые вопросы с ответами ДА/НЕТ. При ответе ДА мы
заплатим 1 рубль, а при ответе НЕТ~— два рубля. Сколько необходимо и достаточно заплатить для отгадывания числа?

\paragraph{Верхняя оценка.} Давайте задавать вопросы так, чтобы отрицательные ответы приносили в два раза больше информации, чем положительные. Тогда за каждый бит информации мы заплатим \(c\log n\) для некоторой константы \(c\). Пусть вопросы будут вида ,,\(x\in T\)?``. Тогда требуется
\[2(\log |X| - \log|X \cap T|) = \log |X| - \log|X\cap\overline T|.\]
Пусть \(|X \cap T| = \alpha|X|\), тогда \(|X\cap\overline T| = (1 - \alpha)|X|\),
т.о. \(\alpha^2 = 1 - \alpha\), \(\alpha=(\sqrt 5 - 1) / 2\). При любом ответе мы заплатим \(c = 1/(-\log \alpha)\approx 1.44\) рублей за бит, а в целом~— \(\log n / (-\log\alpha)\) рублей.

\paragraph{Нижняя оценка.} Применим рассуждение про злонамеренного противника (adversary argument). Пусть противник
выбирает ответ ДА/НЕТ в зависимости от того, какое из двух значение \(1/(\log |X| - \log|X \cap T|)\) и \((2/\log |X| -
\log|X \cap \overline T|)\) больше. При любых \(X,\ T\) одно из этих значений не меньше \(c = 1/(-\log\alpha)\). Таким
образом мы заставляем алгоритм платить не менее \(c\) рублей за бит, а значит любой алгоритм в худшем случае заплатит
\(\lceil c\log n\rceil\) рублей.

\subsection{Применение: упорядочивание камней по весу}
\subsubsection{Верхняя и нижняя оценки для произвольного $N$}
Сколько сравнений нужно сделать для того, чтобы упорядочить \(N\) камней по весу?

\paragraph{Нижняя оценка.} Потребуется \(\lceil\chi(S_N)\rceil = \lceil\log n!\rceil\) сравнений.  

\paragraph{Верхняя оценка.} Будем сортировать вставкой с бинарным поиском места вставки. Количество сравнений:
\[
\lceil\log 2\rceil + \lceil\log 3\rceil +\dotsb+ \lceil\log n\rceil \le \log n! + n - 1 = n\log n + O(n).
\]

\subsubsection{Точные оценки для маленьких $N$}
\begin{exercise}
Сколько нужно взвешиваний, чтобы упорядочить \(N\) камней по весу? 
Найдите точный ответ на этот вопрос для \(N = 2, 3, 4, 5\). Указание: воспользуйтесь жадной стратегией, при которой каждое взвешивание приносит максимум информации.
\end{exercise}

\subsection{Применение: поиск фальшивой монетки}
\begin{itemize}
\item 20 монет, одна фальшивая легче остальных.

Каждое взвешивание даёт не более \(\log 3\) битов. 
Итого \(k\ge\log N/\log 3 = \log_3 N\).

\item 13 монет, одна фальшивая (с неизвестным относительным весом), 3 взвешивания.

Два варианта первого шага:
\begin{itemize}
\item если взвешиваем по 4, то при равенстве нельзя из 5 за два взвешивания найти фальшивую (остаётся 10 исходов),
\item если взвешиваем по 5, то при неравенстве остаётся 10 возможных исходов.
\end{itemize}

\item 15 монет, одна фальшивая, три взвешивания. Не требуется узнавать относительный вес монеты.

Всего исходов \(2\cdot 14 + 1 > 27\), т.к. только в случае трёх равенств мы можем не узнать относительный вес фальшивой монеты.

\item 14 монет, одна фальшивая, три взвешивания. Не требуется узнавать относительный вес монеты.

Всего исходов \(2\cdot 13 + 1 \le 27\), но определить тем не менее нельзя. Аппарата информации по Хартли недостаточно.

\end{itemize}

\subsection{Логика знаний}
В этом разделе мы будем называть множество исходов $A$ множеством \emph{миров}.
Пусть $f$~--- это некоторая функция из $A$ в некоторое множество $I$ (будем воспринимать это как информация о мире).
Нам не важно какие значения принимает $f$, нам будут важны лишь классы эквивалентности, на которые $f$ разбивает $A$:
каждый класс эквивалентности будет состоять из миров $A$ с одинаковым значением $f$.

\begin{example}
    Пусть $A = \{1,2,3,4,5\}$, а $f(x) = x \bmod 3$. Тогда $f$ разбивает $A$ на три класса эквивалентности 
    $\{1,4\}$, $\{2,5\}$ и $\{3\}$.
\end{example}

Пусть $B\subset A$~--- это некоторое \emph{утверждение} о мирах. $B$ \emph{истинно} в мире $x$, если $x\in B$.
В противном случае $B$ \emph{ложно} в $x$. В мире $x$ мы \emph{знаем, что $B$ истинно}, если $y \in B$ для всех 
$y\sim x$.

\begin{example}
    Пусть $A = \{1,2,3,4,5\}$, а $f(x) = x \bmod 3$. Тогда в мирах $1$, $4$ и $3$ мы знаем, 
    что мир меньше $5$.  А в мирах $2$ и $5$~--- не знаем.
\end{example}
\begin{remark}
    ,,Не знаем`` мы будем понимать в смысле ,,не верно, что знаем``.
\end{remark}

К утверждениям о мирах можно применять обычные логические связки: <<И>> (пересечение), <<ИЛИ>> (объединение),
<<НЕ>> (дополнение).

\begin{statement}
    Если в мире $x$ мы знаем $B$, то в мире $x$ мы знаем, что мы знаем $B$.
    Аналогично, если в мире $x$ мы не знаем $B$, то в мире $x$ мы знаем, что не знаем $B$.
\end{statement}

Пусть теперь у нас есть $k$ человек со своими знаниями о мире. 
Они определяют $k$ отношений эквивалентности $\sim_1,\sim_2,\dotsc,\sim_k$ и,
соответственно, $k$ разбиений на классы эквивалентности.

\begin{example}
    Пусть множество миров $A = \{1,2,3,4,5\}$ и есть два человека, Алиса и Боб.
    Алиса знает значения $f_A(x) = x \bmod 3$, а Боб знает $f_B(x) = x\bmod 2$.
    Тогда классы эквивалентности Алисы: $\{1,4\}$, $\{2,5\}$ и $\{3\}$,
    а классы эквивалентности Боба:  $\{1,3,5\}$ и $\{2,4\}$.
    В мире 1 Алиса знает, что мир меньше 5, а Боб не знает. В мире 4 они оба это знают.
    В мире 1 Алиса не знает, что Боря не знает, что мир меньше 5 (действительно, в мире 4,
    который с точки зрения Алисы эквивалентен 1, Боря это знает).
\end{example}

\end{document}
% vim: set tw=120:
