\documentclass[a4paper]{article}
\usepackage{cmap}
\usepackage[T2A]{fontenc}
\usepackage[utf8]{inputenc}
\usepackage[russian]{babel}
\usepackage{amsthm,amsmath,amssymb}
\usepackage{xspace}
\usepackage{fullpage}

\begin{document}

\begin{center}
    \Large Подготовительные задачи по курсу ,,Теория информация``\\
    весна 2019 года
\end{center}

\begin{enumerate}
    \item Сколько вопросов на ДА/НЕТ нужно задать, чтобы определить загаданное число от 1 до \(N\), если 
    \begin{itemize}
        \item вопросы можно задавать адаптивно;
        \item вопросы нужно написать на бумажке заранее.
    \end{itemize}
    Дайте верхнюю и нижнюю оценку.

    \item Сколько вопросов на ДА/НЕТ нужно задать, чтобы определить загаданное число от 1 до \(N\), если загадавшему разрешается один раз солгать и
    \begin{itemize}
        \item вопросы можно задавать адаптивно;
        \item вопросы нужно написать на бумажке заранее.
    \end{itemize}
    Дайте верхнюю и нижнюю оценку.

    
    \item Сколько сравнений нужно сделать для того, чтобы упорядочить \(N\) камней по весу (все камни имеют различный вес)? 
    Дайте верхнюю и нижнюю оценку. Найдите точный ответ на этот вопрос для \(N = 2, 3, 4, 5\). 
    
    \item Дана 81 монетка одного достоинства. Ровно одна из этих монет фальшивая, которая легче остальных.
    За какое минимальное число взвешиваний на чашечных весах можно определить фальшивую монетку?
    
    \item Даны 20 монет, одна из них фальшивая легче остальных.
    За какое минимальное число взвешиваний на чашечных весах можно определить фальшивую монетку?
        
    \item Даны 12 монет, одна из них фальшивая (с неизвестным относительным весом). Можно ли её найти за 3 взвешивания?
               
    \item Даны 13 монет, одна из них фальшивая (с неизвестным относительным весом). Можно ли её найти за 3 взвешивания?
                
    \item Даны 15 монет, одна фальшивая (с неизвестным относительным весом). Можно ли её найти за 3 взвешивания, если не требуется узнавать относительный вес монеты.
        
    \item Даны 14 монет, одна фальшивая (с неизвестным относительным весом). Можно ли её найти за 3 взвешивания, если не требуется узнавать относительный вес монеты.
        
    \item Пусть имеется некоторое неизвестное число от 1 до \(n\) (где \(n\ge2)\).
    Разрешается задавать любые вопросы с ответами ДА/НЕТ. При ответе ДА мы
    заплатим 1 рубль, а при ответе НЕТ~— два рубля. Сколько необходимо и достаточно заплатить для отгадывания числа?
    
\end{enumerate}

\paragraph{Указание.} При решении этих задач попытайтесь в каждом случае найти ответ на вопрос \emph{,,сколько информации даёт вам то или иное действие (ответ на вопрос или взвешивание)``}.
    

\end{document}
