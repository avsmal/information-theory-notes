%& -shell-escape_
\documentclass[12pt]{article}
\usepackage{infocourse}
%opening
\title{,,Теория информации``.\\ Лекция 2 и 3}
\author{А.В. Смаль}

\begin{document}

\maketitle

\section{Комбинаторный подход}
\subsection{Информация по Хартли}
Пусть задано некоторое конечное множество \(A\)~--- \emph{множество исходов}.
\begin{definition}[1928]
Определим \emph{количество информации в \(A\)} как \(\chi(A) = \log_2|A|\) (мы будем измерять количество информации в битах, поэтому все логарифмы будут по основанию \(2\), для байтов основание нужно было бы заменить на \(256\)).
\end{definition}

Если про некоторый \(x\in A\) стало известно, что \(x\in B\), то теперь для идентификации \(x\) нам достаточно \(\chi(A\cap B) = \log |A\cap B|\) битов, т.е. нам сообщили \(\chi(A) - \chi(A\cap B)\) битов информации.

\begin{example}
    Предположим, что мы хотим узнать некоторое неизвестное упорядочение множества $\{\seqn{a}{5}\}$. Нам стало известно,
    что \(a_1>a_2\) или \(a_3>a_4\). Сколько битов информации мы узнали? Множество \(A\) состоит из \(5!\) перестановок,
    множество \(B\)~--- из перестановок, которые удовлетворяют новому условию. Легко проверить, что \(|B| = 90\). Итого
    мы узнали \(\log 120 - \log 90 = \log(4/3)\) битов.
\end{example}

Пусть \(A\subset\bitstr\times\bitstr\). Обозначим через \(\pi_1(A)\) и \(\pi_2(A)\) проекции множества \(A\) на первую и вторую координату соответственно, а \(\chi_1(A) = \log|\pi_1(A)|\) и \(\chi_2(A) = \log|\pi_2(A)|\)~--- их сложность по Хартли.

\begin{theorem} 
\(\chi(A) \le \chi_1(A) + \chi_2(A)\).
\end{theorem}

\begin{definition}
Количество информации в второй компоненте \(A\) при известной первой
\[\chi_{2|1} = \log\left(\max_{a\in\pi_1(A)} |A_a|\right),\]
где $A_a = \{(a, x) \mid x\in \pi_2(A)\}$.
\end{definition}

\begin{theorem} 
\(\chi(A) \le \chi_1(A) + \chi_{2|1}(A)\).
\end{theorem}

\begin{theorem}\label{thm:volume}
Для \(A\subset\bitstr\times\bitstr\times\bitstr\)
\[2\cdot\chi(A) \le \chi_{12}(A) + \chi_{13}(A) + \chi_{23}(A).\]
\end{theorem}
\begin{corollary}
Квадрат объёма трёхмерного тела не превосходит произведение площадей его проекций на координатные плоскости.
\end{corollary}

\begin{statement}
Если \(f: X\to Y\)
\begin{enumerate}
    \item является сюръекцией, то \(\chi(Y)\le \chi(X)\),
    \item является инъекцией, то \(\chi(X)\le \chi(Y)\).
\end{enumerate}
\end{statement}

\subsection{Применение: игра в 10 вопросов}
Сколько вопросов на ДА/НЕТ нужно задать, чтобы определить загаданное число от 1 до \(N\), если (a) можно задавать вопросы адаптивно; (б) вопросы нужно написать на бумажке заранее.

Оценка \(\lceil\log N\rceil \) достигается в обоих случаях, если задавать вопросы про биты двоичного представления загаданного числа.

Докажем нижнюю оценку. Пусть \(A=[N]\). Множество \(Q = \{(\seqn{q}{k})\}\)~--- множество протоколов (ответы на вопросы). 
Можно рассматривать \(A\) и \(Q\) как проекции некоторого множества исходов игры \(S\) на разные координаты. Тогда верны следующие неравенства:
\begin{itemize}
\item \( \chi_Q(S) = \chi(Q) \le \chi_1(Q) + \chi_2(Q) + \dotsb + \chi_k(Q) \le k, \)
\item \( \chi_A(S) = \chi(A) \le \chi(S) \le \chi_Q(S) + \chi_{A|Q}(S) \le k + 0 = k. \)
\end{itemize}
Таким образом получаем, что \(\log N = \chi(A) \le k\).

\subsection{Цена информации}
Пусть имеется некоторое неизвестное число от 1 до \(n\) (где \(n\ge2)\).
Разрешается задавать любые вопросы с ответами ДА/НЕТ. При ответе ДА мы
заплатим 1 рубль, а при ответе НЕТ~— два рубля. Сколько необходимо и достаточно заплатить для отгадывания числа?

\paragraph{Верхняя оценка.} Давайте задавать вопросы так, чтобы отрицательные ответы приносили в два раза больше информации, чем положительные. Тогда за каждый бит информации мы заплатим \(c\log n\) для некоторой константы \(c\). Пусть вопросы будут вида ,,\(x\in T\)?``. Тогда требуется
\[2(\log |X| - \log|X \cap T|) = \log |X| - \log|X\cap\overline T|.\]
Пусть \(|X \cap T| = \alpha|X|\), тогда \(|X\cap\overline T| = (1 - \alpha)|X|\),
т.о. \(\alpha^2 = 1 - \alpha\), \(\alpha=(\sqrt 5 - 1) / 2\). При любом ответе мы заплатим \(c = 1/(-\log \alpha)\approx 1.44\) рублей за бит, а в целом~— \(\log n / (-\log\alpha)\) рублей.

\paragraph{Нижняя оценка.} Применим рассуждение про злонамеренного противника (adversary argument). Пусть противник
выбирает ответ ДА/НЕТ в зависимости от того, какое из двух значение \(1/(\log |X| - \log|X \cap T|)\) и \((2/\log |X| -
\log|X \cap \overline T|)\) больше. При любых \(X,\ T\) одно из этих значений не меньше \(c = 1/(-\log\alpha)\). Таким
образом мы заставляем алгоритм платить не менее \(c\) рублей за бит, а значит любой алгоритм в худшем случае заплатит
\(\lceil c\log n\rceil\) рублей.

\subsection{Применение: упорядочивание камней по весу}
\subsubsection{Верхняя и нижняя оценки для произвольного $N$}
Сколько сравнений нужно сделать для того, чтобы упорядочить \(N\) камней по весу?

\paragraph{Нижняя оценка.} Потребуется \(\lceil\chi(S_N)\rceil = \lceil\log n!\rceil\) сравнений.  

\paragraph{Верхняя оценка.} Будем сортировать вставкой с бинарным поиском места вставки. Количество сравнений:
\[
\lceil\log 2\rceil + \lceil\log 3\rceil +\dotsb+ \lceil\log n\rceil \le \log n! + n - 1 = n\log n + O(n).
\]

\subsubsection{Точные оценки для маленьких $N$}
\begin{exercise}
Сколько нужно взвешиваний, чтобы упорядочить \(N\) камней по весу? 
Найдите точный ответ на этот вопрос для \(N = 2, 3, 4, 5\). Указание: воспользуйтесь жадной стратегией, при которой каждое взвешивание приносит максимум информации.
\end{exercise}

\subsection{Применение: поиск фальшивой монетки}
\begin{itemize}
\item 20 монет, одна фальшивая легче остальных.

Каждое взвешивание даёт не более \(\log 3\) битов. 
Итого \(k\ge\log N/\log 3 = \log_3 N\).

\item 13 монет, одна фальшивая (с неизвестным относительным весом), 3 взвешивания.

Два варианта первого шага:
\begin{itemize}
\item если взвешиваем по 4, то при равенстве нельзя из 5 за два взвешивания найти фальшивую (остаётся 10 исходов),
\item если взвешиваем по 5, то при неравенстве остаётся 10 возможных исходов.
\end{itemize}

\item 15 монет, одна фальшивая, три взвешивания. Не требуется узнавать относительный вес монеты.

Всего исходов \(2\cdot 14 + 1 > 27\), т.к. только в случае трёх равенств мы можем не узнать относительный вес фальшивой монеты.

\item 14 монет, одна фальшивая, три взвешивания. Не требуется узнавать относительный вес монеты.

Всего исходов \(2\cdot 13 + 1 \le 27\), но определить тем не менее нельзя. Аппарата информации по Хартли недостаточно.

\end{itemize}

\subsection{Логика знаний}
В этом разделе мы будем называть множество исходов $A$ множеством \emph{миров}.
Пусть $f$~--- это некоторая функция из $A$ в некоторое множество $I$ (будем воспринимать это как информация о мире).
Нам не важно какие значения принимает $f$, нам будут важны лишь классы эквивалентности, на которые $f$ разбивает $A$:
каждый класс эквивалентности будет состоять из миров $A$ с одинаковым значением $f$.

\begin{example}
    Пусть $A = \{1,2,3,4,5\}$, а $f(x) = x \bmod 3$. Тогда $f$ разбивает $A$ на три класса эквивалентности 
    $\{1,4\}$, $\{2,5\}$ и $\{3\}$.
\end{example}

Пусть $B\subset A$~--- это некоторое \emph{утверждение} о мирах. $B$ \emph{истинно} в мире $x$, если $x\in B$.
В противном случае $B$ \emph{ложно} в $x$. В мире $x$ мы \emph{знаем, что $B$ истинно}, если $y \in B$ для всех 
$y\sim x$.

\begin{example}
    Пусть $A = \{1,2,3,4,5\}$, а $f(x) = x \bmod 3$. Тогда в мирах $1$, $4$ и $3$ мы знаем, 
    что мир меньше $5$.  А в мирах $2$ и $5$~--- не знаем.
\end{example}
\begin{remark}
    ,,Не знаем`` мы будем понимать в смысле ,,не верно, что знаем``.
\end{remark}

К утверждениям о мирах можно применять обычные логические связки: <<И>> (пересечение), <<ИЛИ>> (объединение),
<<НЕ>> (дополнение).

\begin{statement}
    Если в мире $x$ мы знаем $B$, то в мире $x$ мы знаем, что мы знаем $B$.
    Аналогично, если в мире $x$ мы не знаем $B$, то в мире $x$ мы знаем, что не знаем $B$.
\end{statement}

Пусть теперь у нас есть $k$ человек со своими знаниями о мире. 
Они определяют $k$ отношений эквивалентности $\sim_1,\sim_2,\dotsc,\sim_k$ и,
соответственно, $k$ разбиений на классы эквивалентности.

\begin{example}
    Пусть множество миров $A = \{1,2,3,4,5\}$ и есть два человека, Алиса и Боб.
    Алиса знает значения $f_A(x) = x \bmod 3$, а Боб знает $f_B(x) = x\bmod 2$.
    Тогда классы эквивалентности Алисы: $\{1,4\}$, $\{2,5\}$ и $\{3\}$,
    а классы эквивалентности Боба:  $\{1,3,5\}$ и $\{2,4\}$.
    В мире 1 Алиса знает, что мир меньше 5, а Боб не знает. В мире 4 они оба это знают.
    В мире 1 Алиса не знает, что Боря не знает, что мир меньше 5 (действительно, в мире 4,
    который с точки зрения Алисы эквивалентен 1, Боря это знает).
\end{example}

\section{Вероятностный подход}
\subsection{Энтропия Ш\'{е}ннона}

Энтропия Шеннона определяет количество информации \(H(\alpha)\) в распределении
вероятностей для некоторой случайной величины \(\alpha\).  Пусть \(\alpha\)
принимает значения из множества \(\{\seqn{a}{k}\}\) с вероятностями
\(\{\seqn{p}{k}\}\), \(p_i\ge 0 \), \(\sum_ip_i=1\).

Нам бы хотелось, чтобы это определение согласовывалось с определением Хартли, т.е. имеют место следующие ,,граничные условия``:
\begin{itemize}
\item если \(p_1=\dotsb=p_k\), то \(H(\alpha) = \log k\),
\item если \(p_1=1\), \(p_2=\dotsb=p_k=0\), то \(H(\alpha) = 0\).
\end{itemize}
Будем искать \(H(\alpha)\) в виде математического ожидания ,,удивления`` от исхода случайной величины (,,удивление`` зависит от вероятности данного исхода).
\[H(\alpha) = \sum_i p_i\cdot \mathrm{impress}(p_i).\]
Граничные условия однозначно определяют функцию \(\mathrm{impress}(p_i) = \log\frac{1}{p_i} = -\log p_i\).
\begin{definition}[1948]
Энтропия Шеннона случайной величины \(\alpha\) 
\[
H(\alpha) = \sum_{i=1}^k p_i\cdot\log\frac1p_i.
\]
(По непрерывности доопределим \(0\cdot \log\frac10 = 0\).)
\end{definition}

Можно вывести это соотношение из определения Хартли более формально. 
Пусть $W_n$~--- это множество всех слов длины $n$ состоящих из букв $\{\seqn{a}{k}\}$,
где каждая буква $a_i$ встречается ровно $n_i = p_i\cdot n$ раз
(будем считать, что вероятности $p_i$ рациональны, и что множество 
$W_n$ определено только тогда, когда все $n_i$ целые).
Информация по Хартли в $W_n$ 
\[
    \chi(W_n) = \log |W_n| = \log \frac{n!}{n_1!n_2!\dotsb n_k!}.
\]
Это выражение можно оценить при помощи формулы Стирлинга.
\[
    \begin{aligned}
    \chi(W_n)&\ =  \log \frac{\poly(n)\cdot (n/e)^n}
    {\poly(n)\cdot(n_1/e)^{n_1}\cdot(n_1/e)^{n_2}\dotsm(n_k/e)^{n_k}} =\\
    &\ = \log \left(\left(\frac{n}{n_1}\right)^{n_1}\cdot
                    \left(\frac{n}{n_2}\right)^{n_2}\dotsm
                    \left(\frac{n}{n_k}\right)^{n_k}\right) + O(\log n) =\\
    &\ = \log \left(\left(\frac{1}{p_1}\right)^{p_1\cdot n}\cdot
                    \left(\frac{1}{p_2}\right)^{p_2\cdot n}\dotsm
                    \left(\frac{1}{p_k}\right)^{p_k\cdot n}\right) + O(\log n) =\\
    &\ = n \cdot \sum_{i=1}^k p_i \cdot \log\frac{1}{p_i} + O(\log n).
    \end{aligned}
\]
В среднем на один символ приходится $\chi(W_n)/n$ битов информации.
В пределе получаем
\[
    \lim_{n\to \infty}\frac{\chi(W_n)}{n} = \sum_{i=1}^k p_i \cdot \log\frac{1}{p_i} = H(\alpha)
\]
(предел нужно брать по бесконечной подпоследовательности натуральных чисел $n$ таких,
для которых все $\{n_i\}$~--- целые).
 

\begin{lemma}\label{lm:entropy-properties}
Для энтропии Шеннона выполняются следующие соотношения.
\begin{itemize}
\item \(H(\alpha) \ge 0\), причём \(H(\alpha) = 0\) \(\iff\) распределение \(\alpha\) вырождено.

\item \(H(\alpha) \le \log k\), причём \(H(\alpha) = \log k\) \(\iff\) величина \(\alpha\) распределена равномерно.

\end{itemize}
\end{lemma}

Для доказательства нам потребуется следующая теорема.

\begin{theorem}[Неравенство Йенсена]
Пусть функция \(f ( x ) \) является вогнутой на некотором промежутке \(\mathcal {X}\)
и числа \(\seqn{q}{n}>0\) таковы, что \(q_1+\ldots +q_{n}=1\). 
Тогда для любых \(\seqn{x}{n}\) из промежутка \(\mathcal X\) выполняется неравенство:
\[
\sum _{{i=1}}^{{n}}q_{i}f(x_{i})\leq f\left(\sum _{{i=1}}^{{n}}q_{i}x_{i}\right). 
\]
\end{theorem}
\begin{proof}[Доказательство леммы \ref{lm:entropy-properties}]
Первое свойство следует напрямую из определения: каждый член суммы \(H(\alpha)\) неотрицателен и равен нулю только в случае, если \(p_i = 0\) или \(p_i = 1\).

Для доказательства второго неравенства перенесём всё в левую часть и применим неравенство Йенсена:
\[
H(\alpha) - \log k 
= \sum_{i=1}^k p_k\cdot\log\frac{1}{p_i} - \sum_{i=1}^k p_i\cdot\log k 
= \sum_{i=1}^k p_k\cdot\log\frac{1}{p_ik} 
\le \log\left(\sum_{i=1}^k p_i\frac{1}{p_i k}\right) = \log 1 = 0.
\]
\end{proof}
Энтропию совместного распределения пары случайных величин \(\alpha\) и \(\beta\) будем обозначать \(H(\alpha,\beta)\).
\begin{lemma}
Выполняются следующие свойства:
\begin{itemize}
    \item \(H(\alpha, \beta) \le H(\alpha) + H(\beta)\), причём равенство достигается тогда и только тогда, когда случайные величины независимы;
    \item \(H(\alpha) \le H(\alpha, \beta)\), причём равенство достигается тогда и только тогда, когда \(\beta\) полностью определяется значением \(\alpha\), 
    т.е. \(\beta = f(\alpha)\).
\end{itemize}
\end{lemma}
\begin{proof}
Введём обозначения для вероятностей событий совместного распределения вероятностей \((\alpha, \beta)\). Пусть пара \((a_i, b_j)\) имеет вероятность \(p_{i,j}\), событие \([\alpha=a_i]\) имеет вероятность \(p_{i,*} = p_{i,1} + \dotsb + p_{i,n}\), а событие 
\([\beta=b_j]\)~--- вероятность \(p_{*,j} = p_{1,j} + \dotsb + p_{k,j}\). В этих обозначениях неравенство \(H(\alpha, \beta) \le H(\alpha) + H(\beta)\) переписывается как
\[
\sum_{i,j}\cdot p_{i,j}\cdot \log\frac1{p_{i,j}} \le
\sum_{i}\sum_{j} p_{i,j}\cdot \log\frac1{p_{i,*}} +
\sum_{j}\sum_{i} p_{i,j}\cdot \log\frac1{p_{*,j}}.
\]
Перенесём всё в левую часть и применим неравенство Йенсена.
\begin{multline*}
\sum_{i,j}p_{i,j}\cdot \log\frac{p_{i,*}\cdot p_{*,j}}{p_{i,j}} \le
\log\left(\sum_{i,j}p_{i,j}\cdot \frac{p_{i,*}\cdot p_{*,j}}{p_{i,j}}\right) =
\log\left(\sum_{i,j}p_{i,*}\cdot p_{*,j}\right) = \\
= \log \underbrace{\left(\sum_{i}p_{i,*}\right)}_1\cdot
     \underbrace{\left(\sum_{j}p_{*,j}\right)}_1 = 0.
\end{multline*}

Равенство в неравенстве Йенсена для $f(x) = \log(x)$ достигается только, если все точки равны, т.е. 
для любых \(i,j\) \(\frac{p_{i,*}p_{*,j}}{p_{i,j}} = c\) для некоторой константы \(c\). Несложно заметить, что \(c =
1\), т.к. выполняется следующее равенство \(\sum_{i,j} {p_{i,*}p_{*,j}} = c \sum_{i,j}{p_{i,j}}\) в котором обе суммы
равны 1. Таким образом в случае равенства \(\alpha\) и \(\beta\) независимы.

Доказательство второго свойства мы получим как следствие из свойств условной энтропии.
\end{proof}

\begin{definition}\label{def:cond-entropy1}
Энтропия \(\alpha\) при условии \(\beta = b_j\)
\[H(\alpha\mid\beta = b_j) = \sum_i \Pr[\alpha = a_i\mid \beta = b_j]\cdot
    \log\frac{1}{\Pr[\alpha = a_i\mid \beta = b_j]}. \]
\end{definition}

\begin{definition}\label{def:cond-entropy2}
\emph{Условная (относительная) энтропия} \(\alpha\) относительно \(\beta\)
\[H(\alpha\mid\beta) = \sum_j \Pr[\beta = b_j] \cdot H(\alpha\mid \beta = b_j). \]
Другими словами
\[H(\alpha\mid\beta) = \mathop{\mathrm{E}}\limits_{b_j\gets\beta}[H(\alpha\mid \beta = b_j)]. \]
Если подставить определение~\ref{def:cond-entropy1}, то можно получить выражение для условной энтропии через отдельные вероятности событий.
\[H(\alpha\mid\beta) = 
\sum_j \Pr[\beta = b_j]\cdot 
\sum_i \Pr[\alpha = a_i\mid \beta = b_j]\cdot 
\log\frac{1}{\Pr[\alpha = a_i\mid \beta = b_j]}  = 
\sum_{i,j} p_{i,j}\cdot\log\frac{p_{*,j}}{p_{i,j}} . \]
\end{definition}
\begin{lemma}
Условная энтропия обладает следующими свойствами.
\begin{itemize}
    \item \(H(\alpha\mid\beta)\ge 0\).
    \item \(H(\alpha\mid\beta) = 0\) \(\iff\) \(\alpha\) однозначно определяется по \(\beta\). 
    \item \(H(\alpha,\beta) = H(\beta)  + H(\alpha\mid\beta) 
                            = H(\alpha) + H(\beta\mid\alpha)\).
\end{itemize}
\end{lemma}
\begin{proof}
Первое свойство выполняется, т.к. условная энтропия это матожидание неотрицательной случайной величины. Второе свойство объясняется тем, что для любого \(j\) распределение \(\langle\alpha\mid\beta=b_j\rangle\) имеет нулевую энтропию, т.е. распределение вырождено и каждому \(b_j\) соответствует ровно один \(a_i\). Третье свойство следует из следующего равенства.
\[
\sum_{i,j} p_{i,j} \cdot \log\frac{1}{p_{i,j}} =
\sum_{i,j} p_{i,j} \cdot \log\frac{1}{p_{*,j}} +
\sum_{i,j} p_{i,j} \cdot \log\frac{p_{*,j}}{p_{i,j}}.
\]
(Нужна аккуратность, если есть строки, которые состоят из одних нулей, т.е \(p_{*,j} = 0\)~--- такие строки не нужно включать в эти суммы.)
\end{proof}
\begin{corollary} \(H(\alpha,\beta) \ge H(\alpha)\), причём равенство достигается тогда и только тогда, когда \(\beta = f(\alpha) \).
\end{corollary}
\begin{proof}
\(H(\alpha,\beta) - H(\alpha) = H(\beta\mid\alpha) \ge 0\). По второму свойству условной энтропии равенство достигается тогда и только тогда, когда \(\beta = f(\alpha)\).
\end{proof}

\subsection{Взаимная информация}
\begin{definition}
\emph{Информация в \(\alpha\) о величине \(\beta\)} определяется следующим соотношением:
\[
    I(\alpha:\beta) = H(\beta) - H(\beta\mid\alpha).
\]
Эту величину так же \emph{называют взаимной информацией случайных величин \(\alpha\) и \(\beta\)}.
\end{definition}
\begin{lemma}
Для взаимной информации выполняются следующие соотношения.
\begin{enumerate}
\item \(I(\alpha:\beta) \le H(\alpha) \).
\item \(I(\alpha:\beta) \le H(\beta) \).
\item \(I(\alpha:\alpha) = H(\alpha) \).
\item \(I(\alpha:\beta) = I(\beta:\alpha) \).
\item \(I(\alpha:\beta) = H(\alpha) + H(\beta) - H(\alpha,\beta) \).
\end{enumerate}
\end{lemma}
\begin{definition}
    Пусть \(\alpha,\beta,\gamma\)~--- случайные величины. Определим
    \emph{взаимную информацию в \(\alpha\) о \(\beta\) при условии \(\gamma\)}.
    \begin{enumerate}
        \item \(I(\alpha:\beta\mid\gamma) = H(\beta\mid\gamma) -
            H(\beta\mid\alpha,\gamma).\)
        \item \(I(\alpha:\beta\mid\gamma) = \sum_\ell I(\alpha:\beta \mid
            \gamma=c_\ell)\cdot \Pr[\gamma = c_\ell].\)
        \item \(I(\alpha:\beta\mid\gamma) = H(\alpha\mid\gamma) + H(\beta\mid\gamma) -
            H(\alpha,\beta\mid\gamma).\)
        \item \(I(\alpha:\beta\mid\gamma) = H(\alpha,\gamma) + H(\beta,\gamma) -
            H(\alpha,\beta,\gamma) - H(\gamma)\).
    \end{enumerate}
\end{definition}
\begin{lemma} 
    Все определения условной взаимной информации эквивалентны.
\end{lemma}
\begin{proof}
    (3) \(\iff\) (4).
    \[
    (3) = H(\alpha\mid\gamma) + H(\beta\mid\gamma) - H(\alpha,\beta\mid\gamma) =
    H(\alpha,\gamma) - H(\gamma) + H(\beta,\gamma) - H(\gamma) -
    H(\alpha,\beta,\gamma) + H(\gamma).
    \]

\end{proof}

\subsection{Применение: опять о поиске фальшивой монетки}
Теперь у нас достаточно знаний, чтобы доказать, что за три взвешивания нельзя
найти одну фальшивую монету из 14, даже если не нужно определять её
относительный вес.
\begin{proof}
    Предположим, что существует способ найти фальшивую монету за три
    взвешивания. Тогда протокол взвешивания можно представить в виде полного
    троичного дерева, где каждый лист помечен номером монетки, которая оказалась
    фальшивой (у нас как раз ровно \(3^3=27\) исходов). 

    Давайте введём следующее распределение вероятностей \(\alpha\). 
    Пусть монета, номер которой находится в листе, соответствующем трём равенствам
    (такой лист только один), имеет номер \(i\). В нашем распределении
    вероятностей монета с номером \(i\) будет фальшивой с вероятностью \(1/27\).
    Оставшиеся монеты оказываются фальшивыми с вероятностями \(2/27\), причём
    с вероятностью \(1/27\) монета оказывается легче, чем настоящая, и с
    такой же вероятностью она оказывается тяжелее настоящей.
    \[
        H(\alpha) = \log 27 = 3\log 3.
    \]

    Пусть случайные величины \(\beta_1\), \(\beta_2\), \(\beta_3\) соответствуют
    результатам первого, второго и третьего взвешивания соответственно.
    Значение $\alpha$ однозначно определяется после трёх взвешиваний:
    \(
        H(\alpha\mid\beta_1,\beta_2,\beta_3) = 0,
    \)
    а следовательно
    \[
        H(\alpha) \le H(\beta_1,\beta_2,\beta_3) \le H(\beta_1) + H(\beta_2) +
        H(\beta_3) \le 3\log3.    
    \]
    Таким образом каждое взвешивание должно иметь энтропию ровно $\log3$.
    Рассмотрим первое взвешивание. Пусть на чашах весов лежит по $k$ монет.
    Вероятность каждого исхода взвешивания ($<$, $>$, $=$) относительно
    распределения $\alpha$ должна быть ровно $1/3$.
    \[
        \Pr[<] = \frac{k}{27} + \frac{k}{27} = \frac13.
    \]
    Таким образом $2k = 9$, а значит нет такого целого $k$.
\end{proof}

\section{Кодирование}
\subsection{Однозначно декодируемые коды}
\begin{definition}
    Будем называть \emph{кодом} функцию $C:\{\seqn{a}{n}\}\to\bitstr$,
    сопоставляющую буквам некоторого алфавита \emph{кодовые слова}.
    Если любое сообщение, которое получено применением кода $C$, декодируется
    однозначно (т.е. только единственным образом разрезается на образы $C$), 
    то такой код называется \emph{однозначно декодируемым}.
\end{definition}

\begin{definition}
    Код называется \emph{префиксным (беспрефиксным, prefix-free)}, если никакое
    кодовое слово не является префиксом другого кодового слова.
\end{definition}

\begin{theorem}[Неравенство Крафта-Макмилана]\label{thm:mcmill}
    Для любого однозначно декодируемого кода со множеством кодовых слов 
    \(\{\seqn{c}{n}\}\) выполняется следующее неравенство:
    \[
        \sum_{i=1}^{n} 2^{-|c_i|} \le 1.
    \]
\end{theorem}
\begin{lemma}
    Для префиксных кодов верно неравенство Крафта-Макмилана.
\end{lemma}
\begin{proof}
    Рассмотрим дерево префиксного кода и посчитаем суммарную меру поддеревьев,
    которые соответствуют кодовым словам.
\end{proof}

\begin{statement}
    Для префиксных кодов верно и обратное: если есть набор целых чисел
    \(\{\seqn{\ell}{n}\}\),
    удовлетворяющие неравенству Крафта-Макмилана
    \[
        \sum_{i=1}^{n} 2^{-\ell_i} \le 1,
    \]
    то существует префиксный код с кодовыми
    словами \(\{\seqn{c}{n}\}\), где \(|c_i| = \ell_i\).
\end{statement}
\begin{proof}
    Отсортируем $\ell_i$ по возрастанию и будем развешивать их в бесконечном
    двоичном дереве, выбирая каждый раз самый левый свободный узел
    соответствующей меры. Можно заметить, что мы всегда сможем найти такой узел.
\end{proof}
\begin{corollary}
    Для любого однозначно декодируемого кода существует префиксный код с теми же
    длинами кодовых слов.
\end{corollary}

\begin{proof}[Доказательства теоремы~\ref{thm:mcmill}]
    Сопоставим кодовым словам $\{c_i\}$ мономы $\{p_i\}$ от переменных $x$ и $y$ таким 
    образом, что каждый '$0$' в кодовом слове соответствует $x$, а каждая '$1$'~--- $y$:
    \[
        c_i = 0110101 \implies p_i(x,y) = xyyxyxy.
    \]
    Рассмотрим следующее выражение для некоторого $L$.
    \[
        \left( \sum_{i=1}^n p_i(x,y) \right)^L = \sum_{\ell=L}^{\max|c_i| \cdot
        L} M_\ell(x,y),
    \]
    где $M_\ell$ обозначает сумму всех получившихся одночленов степени $\ell$.
    Заметим, что в каждом $M_\ell$ не более $2^\ell$ одночленов: в противном
    случае код не был бы однозначно декодируемым~--- каждый одночлен (без учёта
    коммутативности и ассоциативности) мог получиться не более одного раза.
    
    Теперь рассмотрим значение этого выражения при \(x = y = \frac12\).
    \begin{equation}\label{eq:kraftproof}
        \left( \sum_{i=1}^n p_i\bigl(\sshalf,\sshalf\bigr) \right)^L =
        \sum_{\ell=L}^{\max|c_i| \cdot L} M_\ell\bigl(\sshalf,\sshalf\bigr) \le 
        \sum_{\ell=L}^{\max|c_i| \cdot L} (2^{-\ell} \cdot 2^\ell) \le 
        L\cdot\max|c_i| = O(L).
    \end{equation}

    Предположим теперь, что неравенство Крафта-Макмилана не выполняется, т.e.
    \[
        q = \sum_{i=1}^n p_i(1/2,1/2) = \sum_{i=1}^n 2^{-|c_i|} > 1.
    \]
    Сравнивая это с \eqref{eq:kraftproof} получаем противоречие:
    \(
        q^L = O(L)    
    \) 
    (левая часть растёт экспоненциально, а правая~— линейно).
\end{proof}

Пусть для каждого символа алфавита задана вероятность $p_i$. Нас будут
интересовать самые короткие в среднем коды, т.е. такие, что 
\[
        \sum_{i=1}^n p_i\cdot|c_i| \to \min. 
\]
\begin{theorem}[Шеннон]
    Для любого однозначно декодируемого кода выполняется
    \[
        \sum_{i=1}^n p_i\cdot|c_i|\ge \sum_{i=1}^n p_i\cdot \log\frac1{p_i}. 
    \]
\end{theorem}
\begin{proof}
    Перенесём всё в правую часть и применим неравенство Йенсена:
    \[
    \sum_{i=1}^n p_i\cdot\log\frac{2^{-|c_i|}}{p_i}\le 
    \log\sum_{i=1}^n \left(p_i\frac{2^{-|c_i|}}{p_i}\right) = 
    \log\sum_{i=1}^n 2^{-|c_i|} \le \log 1 = 0. 
    \]
\end{proof}

\begin{theorem}[Шеннон]\label{thm:shannon:optcode}
    Для любого распределения вероятностей $\{\seqn{p}{n}\}$ существует
    однозначно декодируемый/префиксный код $\{\seqn{c}{n}\}$, такой что
    \[
        \sum_{i=1}^n p_i\cdot|c_i|\le \sum_{i=1}^n p_i\cdot \log\frac1{p_i} + 1. 
    \]
\end{theorem}
\begin{remark}
    От '$+1$' в правой части никак не избавиться: например, если у нас только два символа в
    алфавите, то $\sum p_i\cdot|c_i| = 1$, в то время как $\sum
    p_i\log\frac{1}{p_i}$ может быть сколько угодно близко к нулю.
\end{remark}

\begin{proof}
    Покажем, что найдутся \(\{\seqn{c}{n}\}\) такие, что $|c_i| =
    \bigl\lceil\log\frac1p_i \bigr\rceil$. Код существует, т.к. для длин $c_i$ выполняется
    неравенство Крафта-Макмилана:
    \[
        \sum_{i=1}^n 2^{-|c_i|} =  
        \sum_{i=1}^n 2^{-\lceil\log\frac1p_i \rceil} \le  
        \sum_{i=1}^n 2^{-\log\frac1p_i} = 
        \sum_{i=1}^n p_i = 1. 
    \]
    Теперь оценим среднюю длину кода:
    \[
        \sum_{i=1}^n p_i\cdot |c_i| =  
        \sum_{i=1}^n p_i\cdot \bigl\lceil\log{\textstyle\frac1p_i} \bigr\rceil < 
        \sum_{i=1}^n p_i\cdot \bigl(\log{\textstyle\frac1p_i} + 1\bigr) =
        \left(\sum_{i=1}^n p_i\cdot \log{\textstyle\frac1p_i}\right) + 1.
    \]
\end{proof}
\subsection{Код Шеннона-Фано}
Упорядочим вероятности символов по убыванию: $p_1\ge p_2\ge\dotsb\ge p_n$.
Уложим на прямой без пропусков отрезки длиной $p_1$, $p_2$,\ldots, $p_n$
и обозначим $i$-ый отрезок через $S_i$, а их объединение~--- через $S$.
Коды тех букв $a_i$, для которых отрезок $S_i$ попал в левую половину $S$,
будут начинаться с '0', а коды тех букв, для которых отрезок $S_i$ попал
в правую часть $S$~--- с '1'. Центральный отрезок может не попасть
целиком в одну из половин $S$. Если центральный отрезок является первым
или последним, то начнём его код, соответственно, с '0' или '1'.
В противном случае отнесём его в произвольную половину $S$. Далее
применяем эту стратегию отдельно для букв из левой половины $S$ и отдельно для
правой половины $S$. Повторяем так пока не получим уникальные коды для всех
символов.     

\begin{definition}
    Будем называть кодирование, при котором для некоторой константы $c$ и для
    всех $i$ выполняется $|c_i|\le - \log p_i + c$, \emph{сбалансированным}.
\end{definition}

\begin{theorem}[Шеннон]
    Средняя длина кода Шеннона-Фано близка к энтропии, но не обязательно
    оптимальна:
    \[
        \sum_{i=1}^n p_i\cdot |c_i| = H + O(1). 
    \]
\end{theorem}
\subsection{Код Хаффмана}
\begin{definition}
    Будем строить \emph{код Хаффмана} по индукции. При $n = 2$ коды $c_1 =
    \langle0\rangle$, $c_2 = \langle1\rangle$. При $n > 2$ будем предполагать,
    что вероятности упорядочены по убыванию $p_1\ge p_2\ge\dotsb\ge p_n$.
    Заменим символы $a_{n-1}$ и $a_n$ на символ $a_{n-1}'$ с вероятностью
    $p_{n-1}' = p_{n-1} + p_n$. Построим код Хаффмана для $n-1$ символа.
    Для символов $a_{n-1}$ и $a_n$ возьмём коды $c_{n-1} = c_{n-1}'0$ и
    $c_{n} = c_{n-1}'1$.
\end{definition}
\begin{lemma}
    Средняя длина кодового слова для кода Хаффмана оптимальна, т.е. не превосходит средней длины
    любого другого префиксного кода (а значит и любого однозначно
    декодируемого).    
\end{lemma}
\begin{corollary}
    Для кода Хаффмана выполняется неравенство из теоремы Шеннона \ref{thm:shannon:optcode}.
\end{corollary}
\begin{remark}
    На энтропию случайной величины иногда удобно смотреть как на среднюю длину
    кода Хаффмена.
\end{remark}

\subsection{Блоковое кодирование}
Для того, чтобы нивелировать неустранимую '$+1$' в средней длине кода, мы будем
кодировать не отдельные символы, а блоки символов.
Пусть каждый блок состоит из $k$ символов. Пусть случайные величины $\seqn{\alpha}{k}$
распределены как $\alpha$ и соответствуют буквам в блоке. 
\[
    H(\seqn{\alpha}{k}) = \sum_{i=1}^k H(\alpha_i) = k\cdot H(\alpha).
\]
Тогда по теоремам Шеннона получается
следующее ограничение на среднюю длину кода символа в блоке:               
\[
    H(\alpha)\le (\text{средняя длина кода буквы в блоке})\le H(\alpha) +
    \frac{1}{k}.
\]
При кодировании блоков длины 100 мы получаем отклонение от энтропии не
более, чем на 0.01. Однако мы не можем применить код Хаффмена, т.к. на
вход алгоритму его построения нужно было бы передать $n^{100}$ частот
символов.

\subsection{Арифметическое кодирование}
Мы построим код со следующим ограничением на среднюю длину:
\[
    \sum_{i=1}^n p_i\cdot|c_i|\le \sum_{i=1}^n p_i\cdot \log\frac1{p_i} + 2, 
\]
что хуже, чем в теореме Шеннона.

\begin{definition}
    Будем называть полуинтервал \emph{стандартным}, если он имеет вид
    $[0.v0_2, 0.v1_2)$, где $v$~--- это некоторая последовательность битов,
    а числа записаны в двоичной системе счисления. Будем сопоставлять каждому
    стандартному интервалу $[0.v0_2, 0.v1_2)$ код $v$.

    Для первой буквы кода на отрезке [0,1] мы отложим слева направо непересекающиеся интервалы длины
    $p_i$. Пусть первая буква блока~--- это $a_{i_1}$, тогда для второй буквы кода мы внутри интервала
    соответствующего $p_{i_j}$ повторим эту операцию (отложим непересекающиеся интервалы), но длины интервалом 
    будут уже масштабированы с коэффициентом
    $p_i$. Повторим эту операцию $k$ раз. Получившемуся интервалу в качестве его кода 
    сопоставим код наибольшего стандартного интервала, который полностью содержится внутри него.
\end{definition}
\begin{statement}
    В интервале $[a,b)$ всегда найдётся стандартный интервал длины $2^{-k}$, где
    \(
    \frac{b-a}{4}<2^{-k}\le \frac{b-a}{2},
    \)
    т.е. длина кода любого интервала при арифметическом кодировании не
    превосходит $\log \frac{4}{b-a} = \log \frac{1}{p_i} + 2$.
\end{statement}

\begin{remark}
    В случае Марковской цепи можно строить код с соответствующими условными
    вероятностями.
\end{remark}

\end{document}
% vim: set tw=120:
