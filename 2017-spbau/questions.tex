% !TeX spellcheck = ru_RU

\documentclass[12pt]{article}
\usepackage{cmap}
\usepackage[T2A]{fontenc}
\usepackage[utf8]{inputenc}
\usepackage[russian]{babel}
\usepackage{amsthm,amsmath,amssymb}
\usepackage{xspace}
\usepackage{fullpage}

%% CALCULATOR
\usepackage{calc}
%% STYLE
\usepackage[dotinlabels]{titletoc}
\usepackage[small]{titlesec}
%% TITLESEC BUG WORKAROUND %%
\usepackage{etoolbox}
\makeatletter
\patchcmd{\ttlh@hang}{\parindent\z@}{\parindent\z@\leavevmode}{}{}
\patchcmd{\ttlh@hang}{\noindent}{}{}{}
\makeatother
%% END %%
\titlelabel{\thetitle.\quad}
%% Puts "." instead of ":" in captions
\usepackage{ccaption}
\captiondelim{. }

%\usepackage{indentfirst}
%% OTHER

\usepackage[bookmarks=false, colorlinks, unicode, pdfstartview=FitH, pdftex]{hyperref}
\hypersetup{ 
 plainpages=true,
 linkcolor=blue,
 citecolor=red,
 menucolor=blue,
 pdfnewwindow=true
}

\usepackage{tikz}
\usetikzlibrary{positioning,calc}

\newcommand{\bits}{\{0,1\}}
\newcommand{\bitstr}{\bits^*}
\newcommand{\sshalf}{{\textstyle\frac12}}
\newcommand{\seqn}[2]{{#1}_1,{#1}_2,\dotsc,{#1}_{#2}}
\newcommand{\seqin}[3]{{#1}_{{#2}_1},{#1}_{{#2}_2},\dotsc,{#1}_{{#2}_{#3}}}
\newcommand{\IC}{\mathrm{IC}}
\newcommand{\poly}{\mathrm{poly}}
\newcommand{\Nat}{\mathbb{N}}

\DeclareMathOperator{\dom}{dom}
\DeclareMathOperator{\rank}{rank}
\DeclareMathOperator{\rng}{rng}

\theoremstyle{definition}
\newtheorem{definition}{Определение}[section]

\theoremstyle{plain}
\newtheorem{theorem}{Теорема}[section]
\newtheorem{lemma}{Лемма}[section]
\newtheorem{statement}{Утверждение}[section]
\newtheorem{corollary}{Следствие}[section]

\theoremstyle{remark}
\newtheorem{example}{Пример}[section]
\newtheorem{exercise}{Упражнение}[section]
\newtheorem{remark}{Замечание}[section]
\newtheorem{problem}{Задача}[section]

\newenvironment{tasks}{\paragraph{Задачи.}\begin{enumerate}}{\end{enumerate}}

%opening
\title{Вопросы к экзамену по ,,теории информации``}

\begin{document}

\maketitle


\begin{enumerate}
    \item \textbf{Информация по Хартли.} Мотивация и определение. Информация в проекциях.
        Игра в 10 вопросов. Цена информации. Упорядочение камней по весу.
        Поиск фальшивой монетки (разные вариации).
        Логика знаний.

    \item \textbf{Энтропия Шеннона.} Мотивация и определение. Свойства энтропии. Энтропия пары.
        Условная энтропия. Свойства условной энтропии.
	    Взаимная информация. Свойства взаимной информации.
	    Применение энтропии к задаче о поиске фальшивой монетки.
	    
	\item \textbf{Кодирование.} Однозначно декодируемые коды.
	Неравенство Крафта-Макмилана. Префиксные коды. 
	Теоремы Шеннона об однозначно декодируемых кодах.
	Код Шеннона-Фано, код Хаффмана. 

	\item \textbf{Блоковое кодирование с ошибками.}
	Блоковое кодирование. Арифметическое кодирование.
	Теоремы Шеннона о блоковом кодировании с ошибками.
	
	\item \textbf{Свойства распределений.} Энтропийные профили 
	для одного, двух и трёх распределений. Неравенства о тройке распределений (два).
	
	\item \textbf{Криптография.} Теорема Шеннона об энтропии ключа шифрования. Схемы разделения секрета. Пороговая схема Шамира.
	Энтропия секрета существенного участника.
	Нижняя оценка $3/2$ на энтропию ключа участника. Теорема Чирмаза.

	\item \textbf{Коммуникационная сложность.} Верхние оценки на
	коммуникационную сложность функций. Нижние оценки: метод
	размера прямоугольников, метод трудного множества и метод ранга
	матрицы. Сложность $EQ$ и $GE$.
	
	\item \textbf{Связь протоколов и формул.} Теорема Карчмера-Вигдерсона. Внешнее и внутреннее информационное разглашение и их связь. Теорема о связи размера протокола
	и внешнего информационного разглашения. Теорема Храпченко.
	
	\item \textbf{Колмогоровская сложность.} Теорема о существовании
	оптимального способа описания. Определение колмогоровской сложности.
	Свойства колмогоровской сложности.
	Слова большой сложности. Невычислимость колмогоровской сложности.
	Связь колмогоровской сложности и энтропии. 
	
	\item \textbf{Условная колмогоровская сложность и сложность пары.} 
	Условная сложность: существование оптимального способа описания, определение и свойства. Сложность пары. Теорема Колмогорова-Левина.
	
	\item \textbf{Метод несжимаемых объектов.} Теорема об иерархии
	языков, распознаваемых конечными автоматами 
	с несколькими головками.
	
	\item \textbf{Алгоритмическая случайность.} Случайность по Мартину-Лёфу. Свойства. Теорема о том, что случайные последовательности
	случайны по Мартину-Лёфу.
	
	\item \textbf{Применение колмогоровской сложности.} 
	Перенос информации по ленте. Нижняя оценка на копирование
	на одноленточной машине Тьюринга. 
	Алгоритм сложения битовых чисел.
	
	\item \textbf{Локальная лемма Ловаса.} Локальная лемма Ловаса.
	Симметричная локальная лемма Ловаса. Применение локальной леммы Ловаса
	к последовательностям с запрещёнными словами.
	
	\item \textbf{Эффективное доказательство леммы Ловаса}.
	Теорема Мозера-Тардош.
	
\end{enumerate}

\end{document}
% vim: set tw=120:
